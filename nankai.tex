% (The MIT License)
%
% Copyright (c) 2021-2023 Yegor Bugayenko
% Copyright (c) 2021-2023 Zhenyu Zhong
%
% Permission is hereby granted, free of charge, to any person obtaining a copy
% of this software and associated documentation files (the 'Software'), to deal
% in the Software without restriction, including without limitation the rights
% to use, copy, modify, merge, publish, distribute, sublicense, and/or sell
% copies of the Software, and to permit persons to whom the Software is
% furnished to do so, subject to the following conditions:
%
% The above copyright notice and this permission notice shall be included in all
% copies or substantial portions of the Software.
%
% THE SOFTWARE IS PROVIDED 'AS IS', WITHOUT WARRANTY OF ANY KIND, EXPRESS OR
% IMPLIED, INCLUDING BUT NOT LIMITED TO THE WARRANTIES OF MERCHANTABILITY,
% FITNESS FOR A PARTICULAR PURPOSE AND NONINFRINGEMENT. IN NO EVENT SHALL THE
% AUTHORS OR COPYRIGHT HOLDERS BE LIABLE FOR ANY CLAIM, DAMAGES OR OTHER
% LIABILITY, WHETHER IN AN ACTION OF CONTRACT, TORT OR OTHERWISE, ARISING FROM,
% OUT OF OR IN CONNECTION WITH THE SOFTWARE OR THE USE OR OTHER DEALINGS IN THE
% SOFTWARE.

\documentclass{nankai}
\usepackage[nopygments]{ffcode}
\usepackage{href-ul}
\usepackage{listings}

\graphicspath{{figures/}}

\lstdefinestyle{latex}
{
  language=[LaTeX]{TeX},
  texcsstyle=*\color{NankaiPurple},
  basicstyle=\ttfamily,
  moretexcs={subtitle, maketitle, PrintCrumb, columnbreak},
  frame=single
}

\title{\LaTeX{} Class \ff{nankai}}
\subtitle{User's Guide}
\author{Yegor Bugayenko, Zhenyu Zhong}
\security{Public}

\begin{document}
\maketitle

Version: 1.7.0

Date: \today

\setcounter{tocdepth}{2}
\tableofcontents
\newpage

\section{Overview}

The provided class \ff{nankai} helps you design your work documents and presentations, keeping the code short and the style elegant enough for management and technical papers.
To use the class you simply mention its name in the preamble:

\begin{lstlisting}[style=latex]
\documentclass{nankai}
\begin{document}
Hello, world!
\end{document}
\end{lstlisting}

The document rendered from this \LaTeX{} code will look similar to the document you are reading now.
We recommend you to use \ff{latexmk} to compile your \ff{.tex} files to \ff{.pdf}.
The simplest setup will require a few files staying next to your \ff{.tex} file in the same directory (``story'' is the name of your project here):

\begin{lstlisting}[basicstyle=\ttfamily, frame=single]
story\
  .latexmkrc
  .gitignore
  story.tex
\end{lstlisting}

The content of your \ff{.latexmkrc} file would be this:

\begin{lstlisting}[basicstyle=\ttfamily, frame=single]
$pdflatex = 'pdflatex %O --shell-escape %S';
\end{lstlisting}

Make sure the \ff{.gitignore} file lists all the files generated by \ff{pdflatex} during the compilation.
You do not need to commit them to your repository since they are temporary and will be generated again when you compile your document.

In order to compile the document, just say \ff{make} on the command line.

\section{Layout Options}

There are a few class options, provided in square brackets after the \ff{\char`\\documentclass}, which can help you fine-tune the layout of your document:

\begin{itemize}
    \item \ff{landscape}
          makes the document in landscape format, also changing the size of the paper to 16x9 inches (the default page size is \href{https://en.wikipedia.org/wiki/Paper_size}{A4}), making it perfect for presentations.
    \item \ff{slides}
          makes all headers larger, assuming the document is in landscape mode and presented as a slide deck.
    \item \ff{nocover}
          avoid printing the cover images on the first page by the \ff{\char`\\PrintFirstPage} command.
    \item \ff{anonymous}
          removes the name of the author everywhere, including the bottom of the page, where the author's name stays next to the name of the company.
    \item \ff{nobrand}
          avoid mentioning the name of Nankai anywhere in the document, and removes the logo too.
    \item \ff{nosecurity}
          avoids mentioning the level of security at the top right corner of the document and also avoids showing the author's ID where it is usually visible.
    \item \ff{nodate}
          do not show the date and time at the bottom of each page, where they usually are rendered in ISO~8601 format.
    \item \ff{nopaging}
          avoids page numbers at the bottom of each page.
    \item \ff{authordraft}
          prints a big ``It is a draft'' message across each page.
    \item \ff{cn}
          sets the language to simplified Chinese.
    \item \ff{breaks}
          forces all |\section| to start from a new page.
\end{itemize}

\section{Preamble}

In the preamble, you can specify meta-information about the document, such as its title or author's name.
Here is how:

\begin{lstlisting}[style=latex]
\documentclass{nankai}
\title{Making Compression Faster}
\subtitle{Technical Report}
\author{Zhenyu Zhong}
\begin{document}
\maketitle
Hello, world!
\end{document}
\end{lstlisting}

The following meta-commands are defined:

\begin{itemize}
    \item \ff{\char`\\title} is the main title of the document to be used in the text and the PDF document's properties.
    \item \ff{\char`\\subtitle} is the subtitle to be printed under the title.
    \item \ff{\char`\\author} is the author of the document in ``first-name last-name'' format.
    \item \ff{\char`\\id} is the author's internal ID, if applicable.
    \item \ff{\char`\\company} is the name of the company/department the author works for.
    \item \ff{\char`\\security} is the level of security of the document, which is usually printed at the top right corner of it; usual values are ``Public'', ``Internal'', ``Confidential'', or ``Secret''.
\end{itemize}

\section{Custom Commands}

Inside the document body, you can use these commands:

\begin{itemize}
    \item \ff{\char`\\PrintFirstPage\{front-image\}}
          prints the first page of a project charter or a similar landscape document, placing the image \ff{front-image.pdf} on the front (the file should be present in the current dir).
          If you don't have the front image file, just leave the first argument empty.
    \item \ff{\char`\\PrintLastPage\{\}}
          prints the last page of a project charter or a similar landscape document.
    \item \ff{\char`\\PrintThankYouPage\{\}}
          prints the last page with a ``Thank You'' message in the center.
    \item \ff{\char`\\PrintDisclaimer\{\}}
          prints a paragraph at the bottom of the page with a standard disclaimer.
\end{itemize}

\section{Custom Environments}

The following helper environments are defined:

\ff{\char`\\note} is a text box with a colored background that can highlight important information.
Here is an example:

\begin{note}
    This is a note.
\end{note}

\section{Best Practices}

You are free to design your documents any way you want.
However, it will be convenient for you and for your readers if you follow the convention we have for business and technical documents.

The rule of thumb is simple: try \emph{not} to format your documents.
Instead, let the class designed by us do this work for you.
Just type the content without changing the layout, adding colors, fonts, etc.
The less you modify the look and feel, the better your readers perceive your documents.

\subsection{Two Columns}

In the landscape format, it is recommended to use two columns for better text readability.
Here is how:

\begin{lstlisting}[style=latex]
\documentclass{nankai}
\begin{document}
\newpage
\begin{multicols}{2}
\section*{First}
Here goes the first column's content.
\columnbreak
\section*{Second}
Here goes the second column content.
\end{multicols}
\end{document}
\end{lstlisting}

\subsection{Crumbs}

When you need to put many small information pieces into one page, we recommend you to use ``crumbs'':

\begin{lstlisting}[style=latex]
\documentclass{nankai}
\begin{document}
\newpage
\section*{Project Details}
\begin{multicols}{2}
\raggedright
\PrintCrumb{Budget}{\$100K}

\PrintCrumb{Duration}{5 months}
\end{multicols}
\end{document}
\end{lstlisting}

\PrintDisclaimer
\PrintThankYouPage

\end{document}
